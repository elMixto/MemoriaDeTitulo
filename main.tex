\documentclass[spanish, a4paper, 12pt, twoside, openany,final]{book} 
\usepackage{textcomp}
\usepackage[T1]{fontenc, url}
\usepackage[utf8]{inputenc}
\usepackage{titlesec}
\setcounter{secnumdepth}{4}
\usepackage{multirow}                           % Extensión para tablas
\usepackage{minted}                             % Permite darle formato al código que se muestre.
\usepackage{adjustbox}                          % Permite mas versatilidad para cosas como includegraphics.
\usepackage{graphicx}                           % Para poner imágenes
\usepackage{amsmath, amssymb, amsthm}           % paquetes matemáticos
\usepackage{parskip}                            % Elimina la sangría
\urlstyle{sf}                                   % Tipo de url
\usepackage{color}                              % Agrega colores.
\usepackage{subcaption}                         % Permite subtextos en las figuras. Es incompatible con el paquete subfig y subfigure
\usepackage[toc,page]{appendix}                 % Permite el uso de apéndices
\usepackage{chngcntr}                           % Necesario para enumerar las tablas
\counterwithin{table}{section}                  % Numeración de tablas 
\counterwithin{figure}{section}                 % Numeración de figuras
\numberwithin{equation}{section}                % Numeración de ecuaciones
\hyphenpenalty=100000                           % No se dividen las palabras al terminar una línea
\sloppy                                         % Ayuda a que las cosas no se salgan de los márgenes.
\raggedbottom                                   % Hace que el tamaño de las páginas llegue hasta donde llegue el texto, no agrega espacio vertical
\usepackage{xparse,nameref}                     % Ayuda en la creación de nuevos comandos
\usepackage[bottom,hang,flushmargin]{footmisc}  % notas al pie de página están fijas al fondo y sin sangría
\interfootnotelinepenalty=10000                 % Previene que las notas al pie de página se pasen a la siguiente página cuando son muy largas
\usepackage{lipsum}                             % Para crear texto de relleno.

% --------- Editar de aquí en adelante --------

% ----- Apariencia e idioma ----- 
\usepackage[spanish]{babel}                                         % Idioma
\graphicspath{{Images/}{../Images/}}                                % Dónde estarán las imágenes
\usepackage[left=4cm,top=4cm,bottom=2.5cm,right=2.5cm]{geometry}    % Márgenes del documento
\usepackage{setspace}                                               % Permite elegir el interlineado
\linespread{1.3}                                                    % Interlineado de uno y medio. 1.6 es interlineado doble.
\usepackage{microtype}                                              % Permite la modificación de los caracteres.


% ----- Secciones ----- % ESTA PARTE SE UTILIZA EN CASO DE USAR LA CLASE ARTICLE

% \titleformat*{\section}{\LARGE\bfseries}                  % Forma del título de \section 
% \titleformat*{\subsection}{\Large\bfseries}               % Forma del  título de \subsection
% \titleformat*{\subsubsection}{\large\bfseries}            % Forma del  título de \subsubsection 

% Las siguientes tres líneas crean el comando \paragraph con la forma del título correcta.

% \titleformat{\paragraph} 
% {\normalfont\normalsize\bfseries}{\theparagraph}{1em}{}
% \titlespacing*{\paragraph}
% {0pt}{3.25ex plus 1ex minus .2ex}{1.5ex plus .2ex}
%-----------------------------------------------

% ----- Figuras y tablas ----- 
\usepackage{fancyhdr}                           % Permite formatear las cabeceras, pies, enumeración, etc.
\usepackage{subfiles}                           % Para agregar los capítulos que se escriben aparte.
\usepackage{array}                              % Para ordenar texto y ecuaciones.
\usepackage[rightcaption]{sidecap}              % Permite agregar texto lateral
\usepackage{wrapfig}                            % Permite poner figuras con texto al rededor.
\usepackage{float}                              % Permite poner figuras en cualquier lugar.
\usepackage[labelfont=bf]{caption}              % Texto en negrita para descripciones (\caption)
\usepackage{amsmath}
\usepackage{amssymb}
\usepackage[para]{threeparttable}               % Tablas vistosas, mirar antes de utilizar.
\usepackage{url}                                % Permite el uso de enlaces URL.
\usepackage[table,xcdraw,dvipsnames]{xcolor}    % Agranda la cantidad de colores.
\usepackage{makecell}                           % Ayuda en la creación de tablas
\usepackage{hhline}                             % Agranda las opciones de las líneas
\usepackage{textcomp}                           % Símbolo de derechos de autor


% ----- Referencias -----
\usepackage{natbib}                                                     % Ambiente de referencias utilizado.
\bibliographystyle{apalike}                                                 % Estilo de referencias APA.
\def\biblio{\clearpage\bibliographystyle{apalike}\bibliography{References}} % Define el comando \biblio para referencias en subarchivos- NO CAMBIAR


% ----- Cabecera y pies -----
\pagestyle{fancy}                           % Se define el estilo fancy
\fancyhead[RO,LE]{\thepage}                 % Número de página en la izquierda para par y derecha para impar
\fancyhead[RE,LO]{\nouppercase{\rightmark}} % Nombre del capítulo en la derecha para par y la izquierda para impar en la cabecera
%\renewcommand{\headrulewidth}{0pt}         % Cambiar para línea más gruesa
\fancyfoot{}                                % Saca el número de la página abajo.

\fancypagestyle{plain}{                     % Se redefine el estilo automático (plain) para que calce con el resto. En particular la 1ra página de cada capítulo
\fancyhf{}                                  % Elimina la cabecera y los pies
\fancyhead[RO,LE]{\thepage}                 % Número de página en la izquierda para par y derecha para impar
\fancyhead[RE,LO]{\nouppercase{\leftmark}}  % Nombre del capítulo en la derecha para par y la izquierda para impar en la cabecera
%\renewcommand{\headrulewidth}{0pt}         % Cambiar para línea más gruesa
\fancyfoot{}                                % Elimina el número de la página abajo
}

%------------------- Cabecera del Resumen y Agradecimientos--------------

\fancypagestyle{resumen}{                   % Se redefine el estilo resumen para que calce con el resto. 
\fancyhf{}                                  % Elimina la cabecera y los pies
\fancyhead[RO,LE]{\thepage}                 % Número de página en la izquierda para par y derecha para impar
\fancyhead[RE,LO]{\nouppercase{Resumen}}    % Nombre del capítulo en la derecha para par y la izquierda para impar en la cabecera
%\renewcommand{\headrulewidth}{0pt}         % Cambiar para línea más gruesa
\fancyfoot{}                                % Elimina el número de la página abajo
}

\fancypagestyle{abstract}{                  % Se redefine el estilo resumen para que calce con el resto. 
\fancyhf{}                                  % Elimina la cabecera y los pies
\fancyhead[RO,LE]{\thepage}                 % Número de página en la izquierda para par y derecha para impar
\fancyhead[RE,LO]{\nouppercase{Abstract}}   % Nombre del capítulo en la derecha para par y la izquierda para impar en la cabecera
%\renewcommand{\headrulewidth}{0pt}         % Cambiar para línea más gruesa
\fancyfoot{}                                % Elimina el número de la página abajo
}

\fancypagestyle{agradecimientos}{                   % Se redefine el estilo resumen para que calce con el resto. 
\fancyhf{}                                          % Elimina la cabecera y los pies
\fancyhead[RO,LE]{\thepage}                         % Número de página en la izquierda para par y derecha para impar
\fancyhead[RE,LO]{\nouppercase{Agradecimientos}}    % Nombre del capítulo en la derecha para par y la izquierda para impar en la cabecera
%\renewcommand{\headrulewidth}{0pt}                 % Cambiar para línea más gruesa
\fancyfoot{}                                        % Elimina el número de la página abajo
}

% ----- Cabecera de la portada ----- 
\fancypagestyle{frontpage}{             % Se define el estilo frontpage.
	\fancyhf{}                          % Elimina la cabecera y los pies
	\renewcommand{\headrulewidth}{0pt}  % Elimina líneas en cabecera
	\renewcommand{\footrulewidth}{0pt}  % Elimina líneas en pies
	\vspace*{1\baselineskip}
	
 \fancyhead[L]{ \includegraphics[width=0.7in]{escudo_udec.png}\hspace{2cm}}
	\fancyhead[C]{UNIVERSIDAD DE CONCEPCIÓN
	\linebreak FACULTAD DE INGENIERÍA
    \linebreak DEPARTAMENTO DE INGENIERÍA CIVIL INDUSTRIAL}
    \fancyhead[R]{\hspace{1cm}\includegraphics[width=0.7in]{Images/FI_Udec.png}}
	
}

% ----- Enlaces clickeables --------
\usepackage{hyperref}   % Permite que todo el documento sea clickeable.
\newcommand\myshade{85} % Permite la redefinición de colores a gusto del usuario

% Para elegir colores propios mirar los nombres relacionados con dvipsnames, aquí un url con los nombres de dvipsnames: https://www.overleaf.com/learn/latex/Using_colours_in_LaTeX

\colorlet{mylinkcolor}{DarkOrchid}   %Hiperlinks internos
\colorlet{mycitecolor}{YellowOrange} %Citas
\colorlet{myurlcolor}{Aquamarine}    %Urls

% Para dejar el documento sin texto en colores cambiar las tres líneas anteriores a Black.

\hypersetup{  %Define la forma en que se verán las cosas clickeables.
  	linkcolor  = mylinkcolor!\myshade!black,    % Aplica el color definido arriba. En este caso DarkOrchid
  	citecolor  = mycitecolor!\myshade!black,    % Aplica el color definido arriba. En este caso YellowOrange
  	urlcolor   = myurlcolor!\myshade!black,     % Aplica el color definido arriba. En este caso Aquamarine
  	colorlinks = true,                          % Elimina las cajas al rededor de lo clickeable y lo reemplaza por palabras a color.
}


%--------------------------------------------------------------------------------------------------------------------------
% ------------------------------------------ Aquí empieza el documento ----------------------------------------------------
%--------------------------------------------------------------------------------------------------------------------------

\usepackage{tikz}
\begin{document}
\def\biblio{}   % Resetea el comando biblio, de lo contrario una lista de referencias será producida después de cada capítulo
                % resets the biblio command, if not here a new reference list will be produced after every chapter

\begin{titlepage}
	
	\newgeometry{top=1 in, bottom=1 in, left=1 in, right= 1 in} 
	
	\thispagestyle{frontpage}
	
	\begin{center}
		
		\vspace*{4\baselineskip}
		
		
		{\Huge \textbf{UN ALGORITMO BASADO EN MACHINE LEARNING PARA EL PROBLEMA POLINOMIAL ROBUSTO DE LA MOCHILA\\}}
		\vspace*{1.5\baselineskip}
		
		%\large{\textit{subtítulo}}\\
		
		\vspace*{1,5\baselineskip}
		
		\large{\textbf{Por: José Ignacio González Cortés}}\\
		
		\vspace{1,5\baselineskip}
		
		\large{Memoria de titulo presentada a la Facultad de ingeniería de la Universidad de Concepción para optar al título profesional de ingeniero civil industrial} 
		
		\vspace{1,5\baselineskip}
		Octubre 2023\\ %Mes y año de la tesis, solo primera letra del mes en mayúsculas
		Concepción, Chile %Ciudad y país de publicación.
		\vspace{1,5\baselineskip}
		
		\large{\textbf{Profesor Guía: Carlos Contreras Bolton}}\\
		
	\end{center}
	
	\vspace*{4\baselineskip}
	
\end{titlepage}


\vfill
\begin{center}
\begin{figure}
    \centering
    \includegraphics[width=1.1\linewidth]{Images/image.png}
    \caption{Cronograma actual}
    \label{fig:cronograma}
\end{figure}
\end{center}

%----------------Página de derechos de autor: elegir entre a) o b) y borrar/comentar la opción NO utilizada-----------------
\thispagestyle{empty}
\mbox{}                         % Ayuda a bajar el texto
\vfill                          % Deja el texto al fondo
\textcopyright\ 2023, José Ignacio González Cortés \\ % Derechos de autor
%a)
Ninguna parte de esta memoria puede reproducirse o transmitirse bajo ninguna forma o por ningún medio o procedimiento, sin permiso por escrito del autor.\\\\
%b)
Se autoriza la reproducción total o parcial, con fines académicos, por cualquier medio o procedimiento, incluyendo la cita bibliográfica del documento
\vspace{1cm}    % lo separa del fondo
\restoregeometry % Devuelve los márgenes después de la portada


%----------------Página de calificaciones (opcional), descomentar para generar-----------------

% Editar en Otros -> Calificaciones.tex

%%Las calificaciones entregan la misma información de la portada y se le agrega la nota y la firma. Esta página es opcional.
\begin{titlepage}
	
	\newgeometry{top=1 in, bottom=1 in, left=1 in, right= 1 in} 
	
	\thispagestyle{frontpage}
	
	\begin{center}
		
		\vspace*{4\baselineskip}
	
		
		{\Huge \textbf{TÍTULO PRINCIPAL\\}}%No abreviar, no subrayar, no usar comillas. Se escribe completamente en mayúsculas
		        \vspace*{1.5\baselineskip}

		\large{\textit{subtítulo}}\\ %No abreviar, no subrayar, no usar comillas. Solo la priemra letra usa mayúsculas
		
        \vspace*{1,5\baselineskip}

		\large{\textbf{Por: Autor}}\\ %Nombre como aparece en registro académico
		
		\vspace{1,5\baselineskip}
		
		\large{Tesis presentada a la Facultad de Ciencias Físicas y Matemáticas de la Univerisdad de Concepción para optar al grado académico de Magister en Ciencias con Mención en Física.} %Cambiar a grado académico correspondiente. 
		
		\vspace{1,5\baselineskip}
		Diciembre 2019\\ %Mes y año de la tesis, solo primera letra del mes en mayúsculas
		Concepción, Chile %Ciudad y país de publicación.
\vspace{1,5\baselineskip}

		\large{\textbf{Profesor Guía: Nombre}}\\ %Nombre del profesor que dirigió el trabajo, de la comisión informante y otros asesores precedidos por los títulos correctos de la unidad académica.  
		%Al margen derecho del profesor guía se estampa la firma y/o calificación.
		

	\end{center}
	

	
\end{titlepage}         % Genera la pagina de calificaciones del archivo calificaciones.tex
%\restoregeometry                           % Devuelve los márgenes después de la página

%\pagenumbering{gobble}         % Suprime la numeración de páginas
%\thispagestyle{plain}          % suprime el encabezado
%\clearpage\mbox{}\clearpage    % Agrega página en blanco

%----------------Página de dedicatoria (opcional), descomentar para generar ---------------------------------


\thispagestyle{empty}
\mbox{}
\vfill
\hfill \text{Dedicatoria}

\restoregeometry

%-------------------------------------------------




%-----------------Página de agradecimientos (opcional), se incluye normalmente-------------------

% Editar en Otros -> Agradecimientos.

\pagenumbering{roman}                            % Empieza la enumeración romana en minúsculas, para mayúsculas usar Roman.


\newpage
\addcontentsline{toc}{chapter}{AGRADECIMIENTOS}  % Agrega esta sección al índice
\section*{AGRADECIMIENTOS}                       % Debe ir en mayúsculas por reglamento de la UdeC, tiene asterisco para no ser numerada.


\vspace*{2\baselineskip}

\lipsum[4-5] % Texto para mostrar la página, Borrar cuando se escriban los agradecimientos

\vspace*{3\baselineskip}




%-----------------Página de resumen (abstract)-------------

% Si la unidad académica lo requiere, se edita en  Otros -> Resumen.tex . El mismo resumen puede ser incluido en inglés (abstract) en la página siguiente, para agregarlo hay un espacio destinado en el mismo archivo antes mencionado.

\newpage
\addcontentsline{toc}{chapter}{Resumen} % Agrega esta sección al índice
\section*{Resumen}                      % Con asterisco para que no sea numerada.

    \par\vspace*{\fill} % Mueve las palabras clave al final de la página
    \textbf{\textit{Keywords --}} Knapsack Problem %Agregar todas las palabras claves asosciadas con la tesis aquí.
    
    %-----------Si se desea poner el Abstract Des-comentar lo siguiente-----------
    \newpage
    \addcontentsline{toc}{chapter}{Abstract} %Agrega esta sección al índice
    \section*{Abstract}
    
    \par\vspace*{\fill} % Mueve las palabras clave al final de la página
    \textbf{\textit{Keywords --}} Knapsack Problem % Agregar las palabras claves en inglés

%--------------Página de índice.  

%\nocite{*}     % Des-comentar si se desea que TODAS las referencias sean impresas en la lista de referencias, incluyendo las que no fueron finalmente citadas en el texto.

\newpage
{\setstretch{1.0}   % Interlineado de la lista.
\tableofcontents
}

\newpage
{\setstretch{1.0} 
\listoftables}

\newpage
{\setstretch{1.0} 
\listoffigures}


\newpage
\addtocontents{toc}{\protect\setcounter{tocdepth}{4}}   % La profundidad del índice queda en 4, 1.1.1.1
\pagenumbering{arabic}                                  % Comienza la numeración arábiga (números normales)
\setcounter{page}{1}                                    % Comienza el contador de páginas en 1

% A continuación se dejan nombres de diversos capítulos o secciones, para cambiar el nombre del archivo tan solo se debe hacer en la carpeta "capitulos" y luego llamarlos de la forma correcta en "\subfile{Capitulos/nuevonombre}".
% Los nombres de los archivos no pueden llevar tíldes ni espacios para el correcto funcionamiento del compilador, esto no tiene nada que ver con que tengan o no tilde en el documento final.

\chapter{Introducción}
\section{Antecedentes generales}

El problema de la mochila (KP, por sus siglas en inglés, Knapsack Problem) es un problema clásico de la investigación de operaciones, que modela generalmente la necesidad de elegir un conjunto de elementos con costos y beneficios individuales, con una restricción de capacidad máxima, con el fin de maximizar el beneficio. El KP ha sido exhaustivamente estudiado debido a su estructura sencilla, y también debido a que muchos otros problemas de optimización más complejos tienen como un subproblema al KP \citep*{martello_knapsack_1990}.

El problema tiene muchas  variantes, una de las cuales es la versión robusta. El RKP (por sus siglas en inglés, Robust Knapsack Problem) formulado originalmente por \cite{eilon_application_1987} para resolver problemas de asignación de presupuesto con aplicaciones reales, muchos de los parámetros del problema están asociados a incertidumbre. El RKP se plantea para encontrar soluciones que sean factibles para todas las posibles variaciones en los costos de los elementos \citep{monaci_exact_2013}.

Otra variante es el problema polinomial de la mochila (PKP, por sus siglas en inglés, Polynomial Knapsack Problem) que incluye el concepto de sinergias, es decir, que la elección de una o más alternativas específicas otorga un beneficio o costo extra según estas relaciones. EL PKP sirve para modelar sistemas cuyas alternativas presentan conflictos entre ellas, o que cooperan para generar mayor beneficio \citep{baldo_polynomial_2023}. De este problema surge el problema polinomial robusto de la mochila (PRKP, por sus siglas en inglés, Polynomial Robust Knapsack Problem). El PRKP toma en cuenta parámetros inciertos y sinergias polinomiales para modelar problemas de selección de alternativas, que se perjudican o benefician entre sí y además muestran comportamiento estocástico.

Debido a la complejidad espacial del PRKP, se han explorado aplicaciones del problema cuadrático de la mochila (QKP, por sus siglas en inglés, Quadratic Knapsack Problem) \citep{gallo_quadratic_1980} y el problema cúbico de la mochila (CKP, por sus siglas en inglés, Cubic Knapsack Problem) \citep{forrester_strengthening_2022} El QKP presenta sinergias entre dos elementos y ha demostrado ser útil en un gran espectro de aplicaciones como posicionamiento satelital \citep{witzgall_mathematical_1975}, localizaciones de centros de transporte como aeropuertos, ferrocarriles o terminales de carga \citep{rhys_selection_1970}. El CKP es extendido desde el QKP y considera sinergias hasta con tres elementos, además posee aplicaciones como en el problema de satisfacción Max 3-SAT \citep{kofler_penalty_2014}, el problema de selección \citep{gallo_fast_1989}, el problema de alineación de redes \citep{mohammadi_triangular_2017}, y la detección y tratamiento de enfermedades de transmisión sexual \citep{zhao_treatments_2008}.

Por tanto, este trabajo propone la exploración de técnicas avanzadas de machine learning para resolver el PRKP y obtener resultados más eficientes y cercanos a la solución óptima con base en requerimientos de tiempo, memoria y robustez de las soluciones.

\section{Objetivos}
\subsubsection{Objetivo general}
Implementar una heurística basada en machine learning para resolver el PRKP.
\subsubsection{Objetivos específicos}
\begin{itemize}
	\item Revisar la literatura relacionada con problemas de la mochila similares y metodologías aplicables.
	\item Diseñar una heurística basada en machine learning para el PRKP.
	\item Implementar la heurística propuesta basada en machine learning.
	\item Evaluar los resultados y comparar el rendimiento con las metodologías expuestas anteriormente desde la literatura.
\end{itemize}

\section{Revisión de literatura}
El problema polinomial robusto de la mochila es un problema de selección de alternativas, donde, dado un presupuesto, cada alternativa tiene un costo nominal y un costo máximo con el que puede variar. Distintas combinaciones de estas alternativas producen una serie de efectos en el beneficio final. El número de elementos en cada una de estas posibles combinaciones es lo que se considera, el grado de la sinergia polinomial

Existen variedad trabajos enfocados en el QKP y en menor medida para el CKP. Sin embargo, recientemente \cite{baldo_polynomial_2023} ha introducido por primera vez una metodología para resolver el PRKP, utilizando un algoritmo genético y otro algoritmo basado en machine learning. Este último usa un clasificador random forest predice la probabilidad de cada elemento de estar presente en la solución óptima, para decidir basado en una heurística, si considerar cada elemento o no, fijando los elementos con mayor confianza, y resolviendo para el resto de elementos usando gurobi.

Se han realizado revisiones bibliográficas exhaustivas para el KP, como en \cite{kellerer_knapsack_2004} y \cite{pisinger_quadratic_2007} El primero revisa una variedad de definiciones para el problema, así como métodos exactos, algoritmos aproximados, versiones relajadas y descripciones de variaciones comunes del problema en las que se encuentra el QKP. Mientras que \cite{pisinger_quadratic_2007}, se enfoca directamente en el QKP, revisando multitud de enfoques para solucionarlo, descomposiciones y relajaciones lagrangianas, linealizaciones de los parámetros y otras metodologías avanzadas.  Por otro lado, el CKP ha sido abordado por \cite{forrester_strengthening_2022} mejorando las formulaciones lineales clásicas para resolver este problema.

Es interesante revisar enfoques basados en machine learning para abordar este tipo de problemas. \cite{li_novel_2021} ha estudiado el uso de redes neuronales para obtener predicciones de KPs con funciones objetivo no lineales, obteniendo buenos resultados con una estructura basada en teoría de juegos, junto al uso de redes neuronales adversarias.

\cite{rezoug_application_2022} usa distintas técnicas para evaluar las características de los elementos, entre ellos redes neuronales, regresión de procesos gaussianos, random forest y support vector regression. Así, resuelve el problema original, usando solo un subconjunto de los elementos, para luego, mediante el descenso de gradiente y el uso de características de los elementos, decidir cuáles de los anteriormente excluidos, agregar a la solución inicial obtenida. El modelo de machine learning usado para evaluar que elementos son incluidos muestra resultados competitivos con los demás clasificadores e impactos en tiempo computacional insignificantes.

\cite{afshar_state_2020} propuso un algoritmo para generar soluciones para el KP usando un modelo de Deep Reinforcement Learning que selecciona los elementos de forma voraz. El algoritmo propuesto construye las soluciones con base en las decisiones del modelo y genera soluciones con una razón de similitud con el óptimo del 99.9\% usando una arquitectura de A2C con un paso de cuantización de características.

Si bien estos trabajos no se relacionan directamente con la variante del problema propuesto, sí evidencian que las técnicas de Machine learning pueden usarse de forma efectiva para caracterizar y construir soluciones para el PRKP.


\clearpage

\chapter{Metodologia}
    \section{Modelo matemático}
    
    En la formulación de \cite{baldo_polynomial_2023} se describe la formulación del PRKP y realiza una linearización para transformar el problema a un problema de programación lineal (PL) compatible con solvers adecuados como Gurobi o Cplex. Para efectos de esta memoria, no es necesario usar la linearización del problema y se referiŕa a la primera formulación de baldo y se reservará su linearización para uso exclusivo como modelo de PL.
    
    La formulación clásica consiste en un conjunto elementos o alternativas que poseen un beneficio, un costo nominal y un costo máximo asociados, definidos como:
    
    \begin{itemize}
    	\item $I$, El conjunto de elementos posibles, de cardinalidad $N$
    	\item $P_i$, El beneficio asociado al elemento $i$ %P es de profit
    	\item $LC_i$, El coste nominal del elemento $i$    %Lower Cost
    	\item $UC_i$, El coste máximo del elemento $i$     %Upper Cost
    	\item $W$, El presupuesto o coste máximo asociado al problema.
    	\item $S$, El conjunto de sinergias polinomiales.
    \end{itemize}
    
    Así, nuestra variable de decisión es el vector binario definido en \ref{equ:x_def}
    
    \begin{equation}
    	\label{equ:x_def}
    	x_i = \left\{ 
    	\begin{array}{lc}
    		1 & \text{si el elemento $i$ es elegido}\\ \\ 
    		0 &  \text{si el elemento $i$ no es elegido}
    	\end{array} \right.
    \end{equation}
    
    Ahora bien, dada una solución x, algunos elementos elegidos pueden variar en sus costes, con valores entre $LC_i$ y $UP_i$. El máximo número de elementos que puede variar su coste dada una solución se describe por el parámetro $\Gamma$.
    
    Las soluciones robustas encontradas deben tener la propiedad, de que para cualquier combinación posible de costos obtenidos entre LC y UC, no debe superarse el presupuesto. Para esto se asume el peor de los casos para las variaciones, es decir, donde todos los costes que varían son elementos de la solución y además se eligen los $\Gamma$ elementos con mayor variación entre coste nominal y máximo y se varían.
    
    De esta forma se define la variación de costo de un elemento $i$ como:
    
    $$
    \Delta C_i  = UC_i - LC_i
    $$
    
    Y la restricción de presupuesto, como:
    
    \begin{equation}
    	\left( \sum_{i=1}^I LC_i\cdot x_i\right)  + \max \left( \sum_{i=1}^I \Delta C_i\cdot y_i \right) \leq W
    \end{equation}
    
    Donde la variable auxiliar $y_i$ describe si el elemento $x_i$ varía o no, lo que por la formulación está sujeto a:
    
    $$
    \sum_{i=1}^I y_i \leq \Gamma
    $$
    
    \subsubsection{Sinergias polinomiales}
    Las sinergias polinomiales corresponden a beneficios asociados a combinaciones específicas de elementos elegidos para una solución.
    
    Cada sinergia $A \subseteq I$ tiene entonces asociado un beneficio $PS_A$ (Profit Sinergy), y este beneficio se suma, solo si cada elemento de la sinergia está presente en la solución. Los beneficios totales obtenidos por las sinergias se muestran en \ref{total_sinergies_profit} y se entiende de forma comprensiva que, si todos los elementos $i$ en $A$ tienen un $x_i$ con un valor de 1 entonces se suma el beneficio, pero si uno de los elementos no está en la solución, es decir $x_i = 0$, entonces toda la productoria es cero y el beneficio agregado por la sinergia es cero.
    
    \begin{equation}
    	\label{total_sinergies_profit}
    	\sum_{A \in S}\left( PS_A \cdot \prod_{i \in A} x_i \right)
    \end{equation}
    
    
    \subsubsection{Función objetivo}
    Dados los parámetros anteriores, se define la función objetivo \ref{objective_function}
    
    \begin{center}
    	
    	\begin{equation}
    		\max f(x) = \sum_{i=1}^I p_i\cdot x_i + 
    		\sum_{A \in S}\left( PS_A \cdot \prod_{i \in A} x_i \right)
    		- \left( \sum_{i=1}^I LC_i\cdot x_i  + \max  \sum_{i=1}^I \Delta C_i\cdot y_i \right)
    		\label{objective_function}
    	\end{equation}
    \end{center}
    
    
    Sujeto a la restricción de presupuesto
    
    \begin{equation}
    	\sum_{i=1}^I LC_i\cdot x_i  + \max \sum_{i=1}^I \Delta C_i\cdot y_i \leq W        
    \end{equation}
    
    
    \subsection{Complejidad}
    Si bien la complejidad del espacio de búsqueda del KP tradicional es de $O\left( 2^n
    \right)$, la complejidad del problema ha demostrado ser $O\left( nW\right)$ usando técnicas de branch and bound y acercamientos de programación dinámica, no así con esta variante.
    
    La dificultad de trabajar con el PRKP está en su complejidad espacial de $O\left( 2^n\right)$ asociada a las sinergias polinomiales, cada posible combinación de elementos puede tener un beneficio independiente, por lo que cualquier algoritmo que resuelva el problema de forma exacta debe, como mínimo, leer el espacio de búsqueda.
\clearpage
  
\chapter{Metodología}
\section{Metodología propuesta}

Se propone un solver


\begin{figure}[H]
	\centering
	\begin{tikzpicture}[scale=0.5]
		\foreach \x in {3,...,6} {
			\node[shape=circle,draw=black] (Input\x) at (0,\x)  {};
		}
		\foreach \x in {0,...,9} {
			\node[shape=circle,draw=black] (Hidden1\x) at (2,\x)  {};
			\node[shape=circle,draw=black] (Hidden2\x) at (6,\x)  {};
		}
		
		\node[shape=circle,draw=black] (Output) at (9,4.5)  {};
		\node[shape=rectangle,draw=black] (Transformation) at (12,4.5)  {};
		
		
		\foreach \x in {3,...,6}{
			\foreach \y in {0,...,9}{
				\draw (Input\x) to (Hidden1\y);
			}        
		}
		
		\foreach \a in {0,...,9}{
			\foreach \b in {0,...,9}{
				\draw (Hidden1\a) to (Hidden2\b);}
		}
		
		
		\foreach \x in {0,...,9}{
			\draw (Hidden2\x) to (Output);        
		}
		
		
		
		\draw (Output) to (Transformation);
		
		\node at (-3,6) {$feature 1(i)$};
		\node at (-3,5) {$feature 2(i)$};
		\node at (-3,4) {$feature 3(i)$};
		\node at (-3,3) {$feature 4(i)$};
		
		\node at (2,10) {Capa $1$};
		\node at (6,10) {Capa $N$};
		\node at (9,10) {Out};
		\node at (12,10) {$\Tilde{y}_{pred}(i)$};
		\node at (10.5,8) {$\tanh()$};
		
		
	\end{tikzpicture}
	\caption{Estructura general de la red}
\end{figure}




\section{Instancias}

Las instancias reales que se resolverán provienen de un generador de instancias implementado por \cite{baldo_polynomial_2023} cuyas sinergias tienen en su mayoría beneficios de cero. Las instancias con $I$ mayor a mil, tienen un exponencial de sinergias, para $300 \le I \le 1000$, se usa una generación cuadrática de sinergias, mientras que para instancias con menos de 300 ítems, se agregan sinergias de forma lineal.

Para comparar el rendimiento del algoritmo propuesto con los solvers existentes propuestos por \cite{baldo_polynomial_2023}, se usara el mismo conjunto de instancias.

Estas instancias ejecutarán resueltas en un i7-8550U con 4 nucleos 8 hilos, 32GB ram en Linux 6.6.1.
 
\clearpage

\chapter{Discusión}
    
\clearpage

\chapter{Conclusión}
    
\clearpage

\newpage
\renewcommand\refname{Referencias}          % Nombre para la lista de referencias, también se utiliza "Bibliografía"
{\setstretch{1.0}                           % Interlineado de las referencias 
\addcontentsline{toc}{chapter}{Referencias} % Cambia el nombre de la lista de referencias en el índice 
\bibliography{Referencias.bib}              % Agrega las referencias al documento, estas se ubican en el archivo Referencias.bib
}

\newpage
\renewcommand{\appendixpagename}{Apéndices}     % Nombre al inicio.
\addcontentsline{toc}{chapter}{Apéndices}       % Agrega "Apéndices" al índice

\appendix   % Empieza el ambiente de apéndices, desde ahora en adelante los capítulos, secciones, tablas, figuras, etc. vuelven a empezar su numeración

\chapter{Material}                

\clearpage

% Este segundo apéndice tiene información general sobre LaTeX. Está comentado para que no aparezca en tu documento, pero si lo deseas puedes descomentarlo para ver su contenido.

%\chapter{Algunos consejos sobre \LaTeX{}}
%    \subfile{Otros/Consejos}
%\clearpage

% ÉXITO EN TU TESIS

\end{document} 